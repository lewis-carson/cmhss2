% Report template - A4, single-spaced, 12-point Times New Roman, 1-inch margins
\documentclass[12pt, a4paper]{article}

% Use 1-inch margins
\usepackage[margin=1in]{geometry}

% Times font: newtxtext/newtxmath is a Times-like font for pdflatex
\usepackage{newtxtext,newtxmath}

% For exact Times New Roman with XeLaTeX/LuaLaTeX, uncomment the fontspec lines
% \usepackage{fontspec}
% \setmainfont{Times New Roman}

% Single spacing
\usepackage{setspace}
\singlespacing

% Good typography and paragraph spacing
\usepackage{microtype}
\setlength{\parindent}{0.5in}
\setlength{\parskip}{0pt}

% Useful packages
\usepackage{graphicx}
\usepackage{booktabs}
\usepackage{caption}
\usepackage{subcaption}
\usepackage{float}
\usepackage{hyperref}
\usepackage{fancyhdr}
\usepackage{lipsum} % remove or ignore if you don't want filler text

% Page headers/footers
\pagestyle{fancy}
\fancyhf{}
\renewcommand{\headrulewidth}{0pt}
\fancyfoot[C]{\thepage}

% Title metadata
	\title{Correspondence Networks and Linguistic Change among the American Founding Fathers (1706--1836)}
\author{Lewis Carson (dhqq26) \\ Durham University}
\date{\today}

\begin{document}

% Title Page
\begin{titlepage}
    \centering
    \vspace*{2in}
    {\LARGE \textbf{Correspondence Networks and Linguistic Change among the American Founding Fathers (1706--1836)} \\[0.5em]}
	{\large Lewis Carson (dhqq26) \\ Durham University \\[1em]}
	{\large Date: \today \\[2em]}
	\vfill
	% No abstract is required.
	\vfill
\end{titlepage}
\newpage

% 1. Introduction
\section*{Introduction}
\addcontentsline{toc}{section}{Introduction}
% Begin with a title and introduction where you clearly state which model(s)/method(s) are used.
This report presents a computational analysis of the correspondence patterns of the American Founding Fathers across distinct historical periods, ranging from the Colonial era (1706) to the post-Madison presidency (1836). By employing a combination of network analysis and natural language processing (NLP) techniques, specifically Term Frequency-Inverse Document Frequency (TF-IDF), n-gram analysis, and stylometric measures, I aim to uncover how the intellectual, political, and social positions of these figures evolved over time. The study utilises a "distant reading" approach to identify macro-level patterns in communication structure and vocabulary that would be indiscernible through traditional close reading of the massive dataset of over 183,000 letters.

% 2. Problem and Research Question
\section*{Problem and Research Question}
\addcontentsline{toc}{section}{Problem and Research Question}
% Clearly state and motivate your chosen problem and research question(s). 
% Provide sufficient background information to contextualise this problem for the reader.
The primary research problem addresses the challenge of understanding the evolution of political discussion and personal relationships among the Founding Fathers during a century of rapid transformation. With a dataset exceeding 180,000 documents, manual analysis is insufficient to capture the shifting dynamics of influence and language. This report investigates four key questions: (1) How did the vocabulary and prominent topics of correspondence shift across seven distinct historical periods? (2) How did the structure of the correspondence network change, and who emerged as central figures in different eras? (3) Are there observable trends in the formality and complexity of language used? (4) Do these computational findings align with established historical narratives regarding the development of the United States?

To answer these questions, this study employs a multi-modal computational approach. While network analysis can reveal \textit{who} was communicating, it cannot explain \textit{what} they were discussing. Conversely, linguistic analysis can track vocabulary but ignores the structural connections of the correspondents. By integrating Network Analysis with three distinct NLP techniques (TF-IDF, N-grams, and Stylometry), I aim to triangulate the data.

% 3. Data Collection
\section*{Data Collection}
\addcontentsline{toc}{section}{Data Collection}
% Document how your data was collected, and briefly state any relevant libraries or toolkits that were used. 
% Be sure that your dataset is appropriate for the given research question and sufficient for the model(s)/method(s) employed.
Data was collected from the Founders Online archive (https://founders.archives.gov/), a comprehensive digital collection of the papers of major Founding Fathers. The dataset comprises metadata and full text for 183,673 documents spanning from 1706 to 1836. 

The collection process was implemented in Python using the standard \texttt{urllib} library to interface with the Founders Online API. Due to the large volume of data, the following architecture was developed:
\begin{itemize}
    \item \textbf{Metadata download:} First, a lightweight metadata download collected document IDs, dates, authors, and recipients. Second, a content download script fetched the full text for each document.
    \item \textbf{Checkpointing System:} To handle potential network failures during the long download process, a custom checkpointing mechanism (\texttt{download\_checkpoint.json}) was implemented. This allowed the scraper to resume from the last successful download rather than restarting.
    \item \textbf{Storage Format:} Data was stored in JSON Lines (\texttt{.jsonl}) format. Unlike a standard JSON array which requires loading the entire file into memory, JSONL allows for line-by-line processing, which is essential for handling a text corpus of this magnitude (hundreds of megabytes) on standard hardware.
\end{itemize}

\subsection*{Remote Download and Transfer}
Because the corpus is large and the full content download is I/O-heavy and may run for many hours, the initial data download was executed on the university HPC cluster ``Hamilton'' using a Slurm batch job. A small Slurm script (see \texttt{download.slurm}) was used to submit the process; the job requests moderate resources (48-hour wall time, 8GB memory) and runs the \texttt{download.py} script remotely. Running the download on Hamilton has two key benefits: (1) a stable, high-bandwidth connection to the remote API avoiding home-network interruptions; (2) the ability to restart or checkpoint long-running jobs using Slurm's job control.

\subsection*{Reproducibility and Code Archive}
To facilitate reproducibility, the repository includes scripts for data download and analysis. Key points for reproducing the pipeline are:
\begin{itemize}
    \item Dependencies: Core Python packages used include \texttt{nltk}, \texttt{pandas}, \texttt{networkx}, and \texttt{scikit-learn}. The exact versions used are recorded in the project environment (a \texttt{requirements.txt} is recommended in the archive for submission).
    \item Re-running the pipeline (example): First, ensure the dependencies are installed, then run the scripts in order:
    \begin{itemize}
        \item \texttt{python3 download.py} (or submit via Slurm as above)
        \item \texttt{python3 tfidf.py}
        \item \texttt{python3 ngram.py}
        \item \texttt{python3 stylo.py}
        \item \texttt{python3 create\_network.py}
    \end{itemize}
    \item Checkpoints and one-line resumption means long-running tasks can be resumed via the checks recorded in the \texttt{download\_checkpoint.json} file.
\end{itemize}

\subsubsection*{Code Reuse and Attribution}
Some code and resources are reused from open-source packages and online tutorials. In particular, the project uses:\\
\begin{itemize}
    \item \textbf{NLTK} for tokenisation and lemmatisation (\texttt{WordNetLemmatizer}) and for small preprocessing utilities (Bird et al., 2009).\\
    \item \textbf{NetworkX} for constructing and analysing correspondence graphs.\\
    \item \textbf{Pandas} and standard Python libraries for data handling and file I/O.\\
\end{itemize}

% 4. Model Description and Implementation
\section*{Model Description and Implementation}
\addcontentsline{toc}{section}{Model Description and Implementation}
% Describe the computational model(s)/method(s) and your implementation.
The analysis was implemented using Python, leveraging the Natural Language Toolkit (NLTK) for linguistic processing and standard libraries for data manipulation. The dataset was first partitioned into eight distinct historical periods based on key dates: Colonial (pre-1775), Revolutionary War (1775--1783), Confederation (1784--1789), Washington Presidency (1789--1797), Adams Presidency (1797--1801), Jefferson Presidency (1801--1809), Madison Presidency (1809--1817), and Post-Madison (1817--1836).

Text was lowercased, and non-alphabetic characters were removed. I employed the NLTK \texttt{WordNetLemmatizer} to reduce words to their base forms (lemmas). A custom stopword list was created, combining standard English stopwords with common 18th-century epistolary terms (e.g., ``thou'', ``hath'', ``servant'', ``obedient'', ``favour'') to reduce noise and focus on high-information vocabulary.

A directed graph was constructed where nodes represent historical figures (senders and recipients) and edges represent the volume of correspondence between them. The network data was extracted by parsing the metadata of all 183,673 documents. This structure allows for the calculation of centrality metrics and the visualisation of communication density using chord diagrams.

\subsection*{TF-IDF Analysis}
To identify the characteristic vocabulary of each era, I applied Term Frequency-Inverse Document Frequency (TF-IDF).
\begin{itemize}
    \item \textbf{Term Frequency (TF):} The frequency of a term $t$ in period $d$, normalised by the total word count of that period.
    \item \textbf{Inverse Document Frequency (IDF):} Calculated as $\log(N / df(t))$, where $N=8$ (the number of periods) and $df(t)$ is the number of periods containing term $t$.
\end{itemize}
This approach highlights terms that are statistically over-represented in a specific period relative to the entire timeline.

\subsection*{N-gram Analysis}
I extracted bigrams (2-word sequences) and trigrams (3-word sequences) to capture rhetorical patterns and compound nouns (e.g., ``United States'', ``public good''). Unlike TF-IDF, N-gram analysis relied on raw frequency counts within each period to identify the most common phrases used in daily discourse.

\subsection*{Stylometric Analysis}
To measure the complexity and richness of the language, I calculated three key metrics for each period:
\begin{enumerate}
    \item \textbf{Average Sentence Length:} A proxy for syntactic complexity.
    \item \textbf{Average Word Length:} A proxy for lexical sophistication.
    \item \textbf{Yule's K:} A measure of vocabulary richness that is robust to varying text lengths. It is calculated as $K = 10^4 \times \frac{S_2 - S_1}{S_1^2}$, where $S_1$ is the total number of words and $S_2$ is the sum of the squares of the frequencies of each word. A higher $K$ indicates a richer, more varied vocabulary \cite{yule1944}.
\end{enumerate}

\subsection*{Originality and Appropriateness}
This project deliberately integrates network analysis with linguistic techniques (TF-IDF, n-grams, and stylometrics) to interrogate both the structural (who communicates with whom) and linguistic (what topics and styles are used) aspects of Founders' correspondence. The chosen methods are appropriate because they target the research questions: TF-IDF and n-grams detect changing vocabulary and rhetorical patterns, while stylometry captures the evolution of formal style across periods. The network analysis situates those linguistic features in a social structure, enabling historical interpretation beyond text-only analyses.

\begin{itemize}
    \item \textbf{Document Unit for TF-IDF:} To capture period-specific vocabulary, the dataset is aggregated by period - this reduces noise at the single-letter level and sharpens the contrast between eras.
    \item \textbf{Preprocessing Rationale:} Using the WordNet lemmatiser and a custom stopword list aims to minimise noise while preserving content-bearing tokens (e.g., place and person names). I opted to remove non-alphabetic characters to unify variant spellings (e.g., hyphenated forms), accepting the trade-off that metadata sometimes uses abbreviations.
    \item \textbf{Scale and Efficiency:} The choice of JSON Lines and simple streaming preprocessing allows working with a large capture on commodity hardware. Slurm/cluster usage minimises the fragility of downloading and reduces run-time interruptions.
    \item \textbf{Sensitivity and Robustness:} Where appropriate, I cross-validate findings by comparing TF-IDF keywords with high-frequency n-grams and by manual inspection of outliers to identify OCR or metadata bleed.
\end{itemize}

% 5. Results
\section*{Results}
\addcontentsline{toc}{section}{Results}

The computational analysis of the Founding Fathers' correspondence reveals distinct temporal patterns across network structure, vocabulary usage, and stylistic evolution. These findings map closely to the major political and personal phases of the founders' lives, from the colonial era through the early republic to their final years.

\subsection*{Network Analysis}
The visualisation of the correspondence network across eight historical periods (Figures \ref{fig:network_part1} and \ref{fig:network_part2}) illustrates the structural changes in the "Republic of Letters". Quantitative metrics for these networks are provided in Appendix C (Table \ref{tab:network_stats}).

\begin{figure}[H]
    \centering
    \begin{subfigure}[b]{0.45\textwidth}
        \includegraphics[width=\textwidth]{../network_Colonial.png}
        \caption{Colonial (1706--1774)}
    \end{subfigure}
    \hfill
    \begin{subfigure}[b]{0.45\textwidth}
        \includegraphics[width=\textwidth]{../network_Revolutionary_War.png}
        \caption{Revolutionary War (1775--1783)}
    \end{subfigure}
    
    \vspace{1em}
    
    \begin{subfigure}[b]{0.45\textwidth}
        \includegraphics[width=\textwidth]{../network_Confederation.png}
        \caption{Confederation (1784--1788)}
    \end{subfigure}
    \hfill
    \begin{subfigure}[b]{0.45\textwidth}
        \includegraphics[width=\textwidth]{../network_Washington_Presidency.png}
        \caption{Washington Presidency (1789--1797)}
    \end{subfigure}
    \caption{Evolution of the Correspondence Network}
    \label{fig:network_part1}
\end{figure}

\begin{figure}[H]
    \centering
    \begin{subfigure}[b]{0.45\textwidth}
        \includegraphics[width=\textwidth]{../network_Adams_Presidency.png}
        \caption{Adams Presidency (1797--1801)}
    \end{subfigure}
    \hfill
    \begin{subfigure}[b]{0.45\textwidth}
        \includegraphics[width=\textwidth]{../network_Jefferson_Presidency.png}
        \caption{Jefferson Presidency (1801--1809)}
    \end{subfigure}
    
    \vspace{1em}
    
    \begin{subfigure}[b]{0.45\textwidth}
        \includegraphics[width=\textwidth]{../network_Madison_Presidency.png}
        \caption{Madison Presidency (1809--1817)}
    \end{subfigure}
    \hfill
    \begin{subfigure}[b]{0.45\textwidth}
        \includegraphics[width=\textwidth]{../network_Post-Madison.png}
        \caption{Post-Madison (1817--1836)}
    \end{subfigure}
    \caption{Evolution of the Correspondence Network}
    \label{fig:network_part2}
\end{figure}

The Colonial Era graph displays multiple disconnected components. Benjamin Franklin holds the highest degree (922). During the Revolutionary War, the network size more than triples to 5,079 nodes, with Washington as the central node (degree 3,056).

The Confederation Period shows new nodes of influence appearing (e.g., Jefferson and Adams). The Presidential Eras (Washington through Madison) show Washington and Jefferson as the central nodes. Notably, Jefferson's presidency exhibits the highest individual degree centrality in the entire corpus (4,391).

Finally, the Post-Madison network shows key figures like Jefferson remain central (degree 2,477), though the network density decreases.

\subsection*{TF-IDF Analysis}
Term Frequency-Inverse Document Frequency (TF-IDF \cite{salton1988}) analysis identifies the terms with the highest statistical weight in each period (see Appendix A).

\begin{itemize}
    \item \textbf{Colonial (1706--1774):} Top terms include local place names and estate references such as ``doeg'', ``muddy hole'', and ``cravenstreet''.
    \item \textbf{Revolutionary War (1775--1783):} The list is dominated by military locations and names, including ``middlebrook'', ``peekskill'', and ``tyonderoga''.
    \item \textbf{Confederation (1784--1788):} Terms include agricultural references (``plowed'') and names of European diplomats (``calonne'', ``thulemeier'').
    \item \textbf{Washington Presidency (1789--1797):} Key terms include ``assignats'', ``genest'', and ``hamiltonsecy''.
    \item \textbf{Adams Presidency (1797--1801):} Military and diplomatic terms such as ``artillerists'', ``talleyrand'', and ``tousard'' appear frequently.
    \item \textbf{Jefferson Presidency (1801--1809):} Terms include ``natchitoches'', ``turreau'', and ``osage''.
    \item \textbf{Madison Presidency (1809--1817):} ``Napoleon'', ``bonaparte'', and ``merino'' are among the top terms.
    \item \textbf{Post-Madison (1817--1836):} The vocabulary includes academic terms like ``dormitory'', ``bursar'', and ``university virginia''.
\end{itemize}

\begin{figure}[H]
    \centering
    \includegraphics[width=\textwidth]{tfidf_heatmap.png}
    \caption{Heatmap of Top TF-IDF Terms Across Historical Periods}
    \label{fig:tfidf_heatmap}
\end{figure}

\subsection*{Stylometric Analysis}
Table \ref{tab:stylo} in Appendix B presents the average word length, sentence length, and Yule's K (vocabulary richness) for each period.

\begin{figure}[H]
    \centering
    \includegraphics[width=\textwidth]{stylo_metrics.png}
    \caption{Evolution of Stylometric Metrics Across Historical Periods}
    \label{fig:stylo_metrics}
\end{figure}

\subsection*{N-gram Analysis}
Table \ref{tab:bigrams} in Appendix B lists the top 10 bigrams for each historical period.
% 6. Critical Evaluation
\section*{Critical Evaluation}
\addcontentsline{toc}{section}{Critical Evaluation}
% Critically evaluate the model(s)/method(s) and your approach. 
% This should involve critical analysis of the adequacy of the modelling (e.g., what are the assumptions the models rely on, and do they all hold; are there factors or biases that might invalidate conclusions drawn, etc.), as well as comparison with external data and/or published research on the topic.

\subsection*{Methodological Assumptions and Limitations}
The models rely on several key assumptions. First, Network Analysis assumes that the frequency of correspondence equates to the strength of a relationship. This ignores the qualitative nature of the letters; a single long, intimate letter may signify a stronger bond than ten brief administrative notes. Second, TF-IDF assumes that term frequency correlates with thematic importance. However, in 18th-century epistolary style, formulaic politeness (e.g., "your most obedient humble servant") is highly frequent but semantically empty. While I mitigated this with a custom stopword list, some noise inevitably remains.

Furthermore, the dataset is subject to survival bias, as not all historical correspondence has been preserved, and selection bias, as Founders Online focuses on prominent figures ("Great Man" history). As noted in the lecture material regarding n-gram assumptions, OCR errors and metadata artefacts significantly impact results. For instance, the appearance of ``hamiltonsecy'' in the TF-IDF top terms reveals that editorial annotations (e.g., "Hamilton, Secy.") were not fully separated from the body text during the scraping process. Similarly, spelling variations (e.g., ``Doeg Run'' vs. ``Dogue Run'') fragment the counts for single concepts, a common issue in pre-standardised 18th-century English that simple lemmatisation cannot fully resolve.

\subsection*{Comparison with Historical Narratives}
Despite these limitations, the computational results align remarkably well with established historical narratives, such as Gordon Wood's account of the early republic \cite{wood2009}, validating the "distant reading" approach.
\subsection*{Synthesis of Methods}
By employing multiple methods, I can observe how structural power correlates with linguistic change. The Network Analysis reveals a dramatic centralisation of power, with the network size more than tripling during the Revolutionary War (from 1,454 to 5,079 nodes). This structural explosion coincides perfectly with the stylometric data, where average sentence length jumps from 18.8 words in the Colonial period to 26.4 words during the war. This correlation suggests that the requirements of war and statecraft required a more complex, qualified, and precise mode of expression than private business correspondence.

This structural shift is further mirrored in the N-gram analysis. The bigram ``united state'' is virtually absent in the Colonial period — where local terms like ``muddy hole'' and ``doeg run'' dominate — but appears frequently from the Revolutionary War onwards, signalling the rise of a national consciousness.

However, the methods also diverge in revealing ways. While Stylometry shows a relatively constant sentence length after the Revolutionary War spike, the vocabulary richness (Yule's K) peaks specifically during the Washington presidency at 119.5 (the highest value in the corpus). This suggests that the task of inventing the American presidency required an unprecedented expansion of the political vocabulary.

This alignment suggests that while the models may be noisy at the micro-level (individual words), they are robust at the macro-level (historical trends). This multi-modal approach provides a more nuanced view than any single method could offer. For instance, a purely linguistic analysis might interpret the rise of "United States" as a simple change in topic. However, when overlaid with the network data showing extreme centralisation, I can see that this linguistic shift was propagated through a highly controlled hub-and-spoke structure. The language of the nation did not emerge organically from the periphery; it was broadcast from the centre.

\subsection*{Evaluation Methodology and Validation}
To validate claims derived from computational models, several evaluation strategies were used:
\begin{itemize}
    \item \textbf{Manual verification} of top terms and sample letters: a  spot-check of the top TF-IDF terms in each period ensured that important words correspond to the contexts inferred (e.g., military or administrative contexts).
    \item \textbf{Sensitivity analyses}: To test the stability of TF-IDF rankings, I re-ran TF-IDF with and without the custom stopword list and with a minimal lemmatisation pipeline to measure term stability across preprocessing choices.
    \item \textbf{Metadata consistency checks}: Indexes for place names, person names, and editorial notes were checked for bleed-through (e.g., "hamiltonsecy"), and where these occur they were handled by either removal or separate analysis (mentioned in this report).
\end{itemize}

% Discuss what conclusions can be drawn from the model.
\section*{Conclusions}
\addcontentsline{toc}{section}{Conclusions}
This study demonstrates the utility of computational methods in enriching historical research. The analysis confirms that the Founding Fathers' correspondence patterns were not static but evolved dynamically in response to the changing political landscape. I observed a clear trajectory: from the personal and local concerns of the Colonial era, through the urgent military logistics of the Revolutionary War, to the complex administrative statecraft of the early Republic, and finally to a reflective focus on education and legacy in the Post-Madison years.

Specifically, the integration of network analysis with linguistic profiling provided a dual perspective on this evolution. Structurally, the correspondence network transformed from a fragmented collection of regional elites (e.g., the Virginia planters vs. the Boston intellectuals) into a highly centralised "star" topology during the Revolutionary War. This shift represents the visual signature of a command economy of information, driven by the command-and-control necessities of the conflict. This centralisation persisted into the early presidency, with Thomas Jefferson's era exhibiting the highest degree centrality, reflecting an immense personal investment in managing the administration via correspondence. Finally, the Post-Madison network illustrates a "retirement diffusion," reflecting a shift from administrative command to intellectual exchange, as the network density decreases and the structure loosens.

Linguistically, this structural shift was mirrored by a transformation in discourse. TF-IDF and n-gram analyses revealed a sharp pivot from the concrete, agricultural vocabulary of the colonial gentry (the "disappearance of the local") to the abstract, nationalistic terminology of the republic (the "rise of the nation"). The specific vocabulary tracks the nation's history with precision: the military logistics of the Revolution give way to the economic anxieties of the Washington era ("assignats"), the diplomatic tensions of the Adams presidency ("Talleyrand"), and the continental expansion under Jefferson ("Natchitoches"). Finally, the vocabulary turns reflective and academic in the Post-Madison era ("dormitory", "bursar"), mirroring Jefferson's founding of the University of Virginia. The emergence of institutional terms like "head quarter" and "court martial" replaced the personal salutations of the earlier era. Crucially, the stylometric data indicates that this was not merely a change in topic but in cognitive complexity. The 40\% increase in sentence length during the war, coupled with the peak in vocabulary richness (Yule's K) during the Washington presidency, suggests that the intellectual burden of nation-building required a more sophisticated and varied mode of expression than the private correspondence of the previous era. This reveals a compelling dynamic: while the vocabulary became more \textit{standardised} across the network (as evidenced by the rise of shared national bigrams like "United States"), it simultaneously became more \textit{diverse} within individual letters (as shown by the peak in Yule's K). The Founders were not just adopting a few slogans; they were developing a sophisticated technical language of governance that allowed for precise coordination across a rapidly expanding network.

These findings underscore the value of "distant reading" as a complement to traditional historical scholarship. By quantifying the "Republic of Letters," we have provided empirical support for the narrative that the American nation was forged not just on the battlefield, but through a radical restructuring of communication networks and a deliberate expansion of the political lexicon.

Ultimately, this project reveals that the "Republic of Letters" was not merely a metaphor for intellectual exchange, but a tangible, measurable network that physically and linguistically restructured itself to meet the demands of nation-building. The transition from the loose, egalitarian connections of the colonial era to the centralised, hierarchical structures of the war and early presidency mirrors the political evolution of the United States itself. By mapping these changes, I gain a new appreciation for the logistical and intellectual labour required to create a nation — a labour that is preserved in the very syntax and vocabulary of the Founders' correspondence.

\subsection*{Further Work and Reflection}
To address remaining limitations and to deepen interpretation, a number of further steps could refine and extend this project:
\begin{itemize}
    \item \textbf{Formal hypothesis testing:} Explicit testing - for example, whether sentence length distributions differ significantly between periods - would strengthen claims about stylistic change.
    \item \textbf{Improved OCR and spelling normalisation:} Preprocessing improvements could reduce term fragmentation (e.g., Doeg/Dogue) and editorial artefacts (e.g., hamitonsecy). This includes custom spelling unification or editorial cleaning.
    \item \textbf{Authorship attribution and contextualisation:} Applying supervised classification to determine if stylistic changes are driven by authorship differences or by broader social changes would refine causal claims.
\end{itemize}

\newpage
\section*{References}
\addcontentsline{toc}{section}{References}
% Use BibTeX, BibLaTeX, or manual bibliography here. Example below is manual entries.
\begin{thebibliography}{9}
\bibitem{salton1988} Salton, G., \& Buckley, C. (1988). Term-weighting approaches in automatic text retrieval. Information Processing \& Management, 24(5), 513--523.
\bibitem{wood2009} Wood, G. S. (2009). Empire of Liberty: A History of the Early Republic, 1789-1815. Oxford University Press.
\bibitem{blei2003} Blei, D.M., Ng, A.Y., \& Jordan, M.I. (2003). Latent Dirichlet Allocation. Journal of Machine Learning Research, 3, 993--1022.
\bibitem{yule1944} Yule, G. U. (1944). The Statistical Study of Literary Vocabulary. (Cambridge University Press).
\bibitem{bird2009} Bird, S., Klein, E., \& Loper, E. (2009). Natural Language Processing with Python. O'Reilly Media.
\bibitem{hagberg2008} Hagberg, A., Schult, D., \& Swart, P. (2008). Exploring network structure, dynamics, and function using NetworkX. In Proceedings of the 7th Python in Science Conference (SciPy2008).
\bibitem{nltk} Bird, S., Loper, E., \& Klein, E. (2009). Natural Language Toolkit (NLTK). http://www.nltk.org
\end{thebibliography}

\newpage
\appendix
\section{TF-IDF Results}
\begin{table}[H]
\centering
\caption{Top 10 TF-IDF Terms by Historical Period}
\label{tab:tfidf}
\resizebox{\textwidth}{!}{%
\begin{tabular}{@{}llllllll@{}}
\toprule
\textbf{Colonial} & \textbf{Rev. War} & \textbf{Confederation} & \textbf{Washington} & \textbf{Adams} & \textbf{Jefferson} & \textbf{Madison} & \textbf{Post-Madison} \\ \midrule
doeg & artaud & dogue & hamiltonsecy & subdistricts & pichon & napoleon & dormitory \\
ageneral & middlebrook & williamos & ustates & artillerists & cevallos & batture & dunglison \\
eund & zullen & deral & assignats & mchenry & turreau & merino & montezillo \\
dismd & tyonderoga & mansn & hamiltonsecretary & dittovice & reibelt & correa & bursar \\
patd & sartine & sjl & jaudenes & rivardi & osage & gothenburg & montpr \\
empld & eenige & tabacs & ischem & talleyrand & yrujo & ticknor & brockenbrough \\
dind & peekskill & shipton & chiappe & bewell & ustates & bonaparte & raggi \\
waterson & pompton & deslon & clingman & dittojohn & polygraph & bankhead & capitels \\
cravenstreet & hazen & grd & whitting & subdistrict & gallatin & nonag & ticknor \\
magowan & billingsport & doradour & genest & tousard & gavino & dallas & bonnycastle \\ \bottomrule
\end{tabular}%
}
\end{table}

\section{N-gram Results}
\begin{table}[H]
\centering
\caption{Top 10 Bigrams by Historical Period}
\label{tab:bigrams}
\resizebox{\textwidth}{!}{%
\begin{tabular}{@{}llllllll@{}}
\toprule
\textbf{Colonial} & \textbf{Rev. War} & \textbf{Confederation} & \textbf{Washington} & \textbf{Adams} & \textbf{Jefferson} & \textbf{Madison} & \textbf{Post-Madison} \\ \midrule
new york & head quarter & united state & united state & united state & united state & united state & united state \\
home day & new york & new york & new york & new york & new york & new york & new york \\
little wind & united state & noon night & secretary treasury & secretary war & new orleans & john adam & john adam \\
benjamin franklin & respect excellency & morning noon & secretary state & mount vernon & department state & respect jefferson & james madison \\
george washington & que vous & muddy hole & president united & john adam & james madison & esteem respect & respect jefferson \\
every thing & every thing & mount vernon & house representative & major general & respect jefferson & accept assurance & accept assurance \\
went away & beg leave & dogue run & obedt servt & general hamilton & beg leave & every thing & esteem respect \\
muddy hole & servt washington & thermometer morning & servt jefferson & obedt servt & every thing & jefferson monticello & jefferson monticello \\
mount vernon & obedt servt & que vous & treasury department & war department & president united & james madison & thomas jefferson \\
fort cumberland & west point & every thing & every thing & president united & secretary state & beg leave & university virginia \\ \bottomrule
\end{tabular}%
}
\end{table}

\section{Stylometric Results}
\begin{table}[H]
\centering
\caption{Stylometric Analysis by Period}
\label{tab:stylo}
\begin{tabular}{@{}lcccc@{}}
\toprule
\textbf{Period} & \textbf{Letters} & \textbf{Avg Word Len} & \textbf{Avg Sent Len} & \textbf{Yule's K (Richness)} \\ \midrule
Colonial & 16,151 & 4.39 & 18.81 & 93.11 \\
Revolutionary War & 48,188 & 4.46 & 26.43 & 96.38 \\
Confederation & 17,810 & 4.43 & 21.72 & 97.44 \\
Washington Presidency & 27,156 & 4.51 & 23.02 & 119.51 \\
Adams Presidency & 13,570 & 4.49 & 23.55 & 109.19 \\
Jefferson Presidency & 29,499 & 4.51 & 24.02 & 110.65 \\
Madison Presidency & 15,477 & 4.48 & 26.53 & 106.81 \\
Post-Madison & 15,445 & 4.47 & 26.14 & 109.56 \\ \bottomrule
\end{tabular}
\end{table}

\section{Network Statistics}
\begin{table}[H]
\centering
\caption{Network Statistics by Historical Period}
\label{tab:network_stats}
\resizebox{\textwidth}{!}{%
\begin{tabular}{@{}lrrrl@{}}
\toprule
\textbf{Period} & \textbf{Nodes} & \textbf{Edges} & \textbf{Density} & \textbf{Most Central Figure (Degree)} \\ \midrule
Colonial & 1,454 & 2,211 & 0.0011 & Benjamin Franklin (922) \\
Revolutionary War & 5,079 & 9,755 & 0.0004 & George Washington (3,056) \\
Confederation & 2,396 & 4,454 & 0.0008 & George Washington (1,343) \\
Washington Presidency & 3,505 & 6,196 & 0.0005 & George Washington (2,398) \\
Adams Presidency & 1,994 & 3,388 & 0.0009 & John Adams (1,081) \\
Jefferson Presidency & 4,764 & 7,232 & 0.0003 & Thomas Jefferson (4,391) \\
Madison Presidency & 3,268 & 4,943 & 0.0005 & James Madison (2,217) \\
Post-Madison & 2,815 & 5,346 & 0.0007 & Thomas Jefferson (2,477) \\ \bottomrule
\end{tabular}%
}
\end{table}

\end{document}
