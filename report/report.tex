% Report template - A4, single-spaced, 12-point Times New Roman, 1-inch margins
\documentclass[12pt, a4paper]{article}

% Use 1-inch margins
\usepackage[margin=1in]{geometry}

% Times font: newtxtext/newtxmath is a Times-like font for pdflatex
\usepackage{newtxtext,newtxmath}

% For exact Times New Roman with XeLaTeX/LuaLaTeX, uncomment the fontspec lines
% \usepackage{fontspec}
% \setmainfont{Times New Roman}

% Single spacing
\usepackage{setspace}
\singlespacing

% Good typography and paragraph spacing
\usepackage{microtype}
\setlength{\parindent}{0.5in}
\setlength{\parskip}{0pt}

% Useful packages
\usepackage{graphicx}
\usepackage{booktabs}
\usepackage{caption}
\usepackage{subcaption}
\usepackage{float}
\usepackage{hyperref}
\usepackage{fancyhdr}
\usepackage{lipsum} % remove or ignore if you don't want filler text

% Page headers/footers
\pagestyle{fancy}
\fancyhf{}
\renewcommand{\headrulewidth}{0pt}
\fancyfoot[C]{\thepage}

% Title metadata
\title{Report Title}
\author{Author Name \\ Institution}
\date{\today}

\begin{document}

% Title Page
\begin{titlepage}
	\centering
	\vspace*{2in}
	{\LARGE \textbf{Report Title} \\[0.5em]}
	{\large Author Name \\ Institution \\[1em]}
	{\large Date: \today \\[2em]}
	\vfill
	% No abstract is required.
	\vfill
\end{titlepage}
\newpage

% 1. Introduction
\section*{Introduction}
\addcontentsline{toc}{section}{Introduction}
% Begin with a title and introduction where you clearly state which model(s)/method(s) are used.
This report presents a computational analysis of the correspondence patterns of the American Founding Fathers across distinct historical periods, ranging from the Colonial era (1706) to the post-Madison presidency (1836). By employing a combination of network analysis and natural language processing (NLP) techniques, specifically Term Frequency-Inverse Document Frequency (TF-IDF), n-gram analysis, and stylometric measures, we aim to uncover how the intellectual, political, and social preoccupations of these figures evolved over time. The study utilizes a "distant reading" approach to identify macro-level patterns in communication structure and vocabulary that would be indiscernible through traditional close reading of the massive corpus of over 183,000 letters. The findings reveal a dramatic transformation from personal and local concerns to complex administrative statecraft, and finally to a reflective focus on education and legacy.

% 2. Problem and Research Question
\section*{Problem and Research Question}
\addcontentsline{toc}{section}{Problem and Research Question}
% Clearly state and motivate your chosen problem and research question(s). 
% Provide sufficient background information to contextualize this problem for the reader.
The primary research problem addresses the challenge of understanding the evolution of political discourse and personal relationships among the Founding Fathers during a century of profound transformation. With a corpus exceeding 180,000 documents, manual analysis is insufficient to capture the shifting dynamics of influence and language. This research investigates four key questions: (1) How did the vocabulary and prominent topics of correspondence shift across seven distinct historical periods? (2) How did the structure of the correspondence network change, and who emerged as central figures in different eras? (3) Are there observable trends in the formality and complexity of language used? (4) Do these computational findings align with established historical narratives regarding the development of the American nation?

\subsection*{Methodological Rationale}
To answer these questions, this study employs a multi-modal computational approach. While network analysis can reveal \textit{who} was communicating, it cannot explain \textit{what} they were discussing. Conversely, linguistic analysis can track vocabulary but ignores the structural power dynamics of the correspondents. By integrating Network Analysis with three distinct NLP techniques (TF-IDF, N-grams, and Stylometry), we aim to triangulate the data, providing a holistic view of the "Republic of Letters" that connects social structure with intellectual content.

% 3. Data Collection
\section*{Data Collection}
\addcontentsline{toc}{section}{Data Collection}
% Document how your data was collected, and briefly state any relevant libraries or toolkits that were used. 
% Be sure that your dataset is appropriate for the given research question and sufficient for the model(s)/method(s) employed.
Data was collected from the Founders Online archive (https://founders.archives.gov/), a comprehensive digital collection of the papers of major Founding Fathers. The dataset comprises metadata and full text for 183,673 documents spanning from 1706 to 1836. 

The collection process was implemented in Python using the standard \texttt{urllib} library to interface with the Founders Online API. Due to the large volume of data, a robust scraping architecture was developed:
\begin{itemize}
    \item \textbf{Two-Stage Harvesting:} First, a lightweight metadata harvest collected document IDs, dates, authors, and recipients. Second, a content download script fetched the full text for each document.
    \item \textbf{Checkpointing System:} To handle potential network failures during the long download process, a custom checkpointing mechanism (\texttt{download\_checkpoint.json}) was implemented. This allowed the scraper to resume from the last successful download rather than restarting, ensuring data integrity and efficiency.
    \item \textbf{Storage Format:} Data was stored in JSON Lines (\texttt{.jsonl}) format. Unlike a standard JSON array which requires loading the entire file into memory, JSONL allows for line-by-line processing, which is essential for handling a text corpus of this magnitude (hundreds of megabytes) on standard hardware.
\end{itemize}
The data was subsequently partitioned into eight historical periods (e.g., Colonial, Revolutionary War, Washington Presidency) to facilitate temporal analysis. The scale and rich metadata of this dataset make it highly appropriate for both network and linguistic analysis.

\subsection*{Remote Harvesting and Transfer}
Because the corpus is large and the full content download is I/O-heavy and may run for many hours, the initial data harvest was executed on the university HPC cluster ``Hamilton'' using a Slurm batch job. A small Slurm script (see \texttt{download.slurm}) was used to submit the process; the job requests moderate resources (48-hour wall time, 8GB memory) and runs the \texttt{download.py} script remotely. Running the harvest on Hamilton has two key benefits: (1) a stable, high-bandwidth connection to the remote API avoiding home-network interruptions; (2) the ability to restart or checkpoint long-running jobs using Slurm's job control.

After the remote harvest completed, the resulting raw files (\texttt{letters.jsonl}, \texttt{founders-online-metadata.json}) were securely copied from the HPC filesystem to the local development machine using \texttt{scp} for subsequent processing and analysis. Using the cluster for heavy IO and the local machine for interactive analysis provided an efficient and robust workflow for handling the dataset.

Typical commands used (with the HPC host and path adapted to local cluster policies) included:
\begin{itemize}
    \item Submitting the Slurm job: \texttt{sbatch download.slurm}
    \item Securely copying data to the local machine: \texttt{scp user@hamilton:/scratch/user/letters.jsonl .}
\end{itemize}

\subsection*{Reproducibility and Code Archive}
To facilitate reproducibility, the repository includes scripts for data download and analysis. Key points for reproducing the pipeline are:
\begin{itemize}
    \item Dependencies: Core Python packages used include \texttt{nltk}, \texttt{pandas}, \texttt{networkx}, and \texttt{scikit-learn}. The exact versions used are recorded in the project environment (a \texttt{requirements.txt} is recommended in the archive for submission).
    \item Re-running the pipeline (example): First, ensure the dependencies are installed, then run the scripts in order:
    \begin{itemize}
        \item \texttt{python3 download.py} (or submit via Slurm as above)
        \item \texttt{python3 tfidf.py}
        \item \texttt{python3 ngram.py}
        \item \texttt{python3 stylo.py}
        \item \texttt{python3 create\_network.py}
    \end{itemize}
    \item Checkpoints and one-line resumption means long-running tasks can be resumed via the checks recorded in the \texttt{download\_checkpoint.json} file.
    \item Data handling: Large files are not included in the archive; instructions for downloading via the Founders Online API or running the Slurm pipeline on Hamilton are provided in the project README. A prebuilt archive (if larger than 10MB per assignment policy) should be obtained via the documented script rather than bundling the data directly with the code.
    \item The archive submitted for assessment should include a \texttt{README.txt} or \texttt{README.md} with explicit instructions for installation and replication: package installation (e.g., \texttt{pip install -r requirements.txt}), commands to run the pipeline, and notes on the expected output files and runtime.
\end{itemize}

\subsubsection*{Code Reuse and Attribution}
Some code and resources are reused from open-source packages and online tutorials. In particular, the project uses:\\
\begin{itemize}
    \item \textbf{NLTK} for tokenization and lemmatization (\texttt{WordNetLemmatizer}) and for small preprocessing utilities (Bird et al., 2009).\\
    \item \textbf{NetworkX} for constructing and analyzing correspondence graphs.\\
    \item \textbf{Pandas} and standard Python libraries for data handling and file I/O.\\
\end{itemize}
Where utility code or patterns were adopted (e.g., a JSON streaming read pattern), these are acknowledged in the code and in the archive README, together with the relevant URLs for the original sources. The original contributions of this project are the bespoke preprocessing pipeline for historical letters, the period aggregation for TF-IDF, and the combined multi-method analysis and evaluation strategy.

% 4. Model Description and Implementation
\section*{Model Description and Implementation}
\addcontentsline{toc}{section}{Model Description and Implementation}
% Describe the computational model(s)/method(s) and your implementation.
The analysis was implemented using Python, leveraging the Natural Language Toolkit (NLTK) for linguistic processing and standard libraries for data manipulation. The corpus was first partitioned into eight distinct historical periods based on key dates: Colonial (pre-1775), Revolutionary War (1775--1783), Confederation (1784--1789), Washington Presidency (1789--1797), Adams Presidency (1797--1801), Jefferson Presidency (1801--1809), Madison Presidency (1809--1817), and Post-Madison (1817--1836).

\subsection*{Data Preprocessing}
For all NLP tasks, the text underwent a rigorous preprocessing pipeline. Text was lowercased, and non-alphabetic characters were removed. We employed the NLTK \texttt{WordNetLemmatizer} to reduce words to their base forms (lemmas). A custom stopword list was created, combining standard English stopwords with common 18th-century epistolary terms (e.g., ``thou'', ``hath'', ``servant'', ``obedient'', ``favour'') to reduce noise and focus on content-bearing vocabulary.

\subsection*{Network Analysis}
A directed graph was constructed where nodes represent historical figures (senders and recipients) and edges represent the volume of correspondence between them. The network data was extracted by parsing the metadata of all 183,673 documents. This structure allows for the calculation of centrality metrics and the visualization of communication density using chord diagrams.

\subsection*{TF-IDF Analysis}
To identify the characteristic vocabulary of each era, we applied Term Frequency-Inverse Document Frequency (TF-IDF). Uniquely, we defined a ``document'' not as an individual letter, but as the \textit{aggregated text of all letters within a single historical period}.
\begin{itemize}
    \item \textbf{Term Frequency (TF):} The frequency of a term $t$ in period $d$, normalized by the total word count of that period.
    \item \textbf{Inverse Document Frequency (IDF):} Calculated as $\log(N / df(t))$, where $N=8$ (the number of periods) and $df(t)$ is the number of periods containing term $t$.
\end{itemize}
This approach highlights terms that are statistically over-represented in a specific period relative to the entire timeline.

\subsection*{N-gram Analysis}
We extracted bigrams (2-word sequences) and trigrams (3-word sequences) to capture rhetorical patterns and compound nouns (e.g., ``United States'', ``public good''). Unlike TF-IDF, N-gram analysis relied on raw frequency counts within each period to identify the most common phrases used in daily discourse.

\subsection*{Stylometric Analysis}
To measure the complexity and richness of the language, we calculated three key metrics for each period:
\begin{enumerate}
    \item \textbf{Average Sentence Length:} A proxy for syntactic complexity.
    \item \textbf{Average Word Length:} A proxy for lexical sophistication.
    \item \textbf{Yule's K:} A measure of vocabulary richness that is robust to varying text lengths. It is calculated as $K = 10^4 \times \frac{S_2 - S_1}{S_1^2}$, where $S_1$ is the total number of words and $S_2$ is the sum of the squares of the frequencies of each word. A higher $K$ indicates a richer, more varied vocabulary \cite{yule1944}.
\end{enumerate}

\subsection*{Originality and Appropriateness}
This project deliberately integrates network analysis with linguistic techniques (TF-IDF, n-grams, and stylometrics) to interrogate both the structural (who communicates with whom) and linguistic (what topics and styles are used) aspects of Founders' correspondence. The originality lies not in the novelty of individual algorithms—these are well-known in computational humanities—but in their combined application at scale to a large historical corpus with careful attention to periodization. The chosen methods are appropriate because they target the research questions: TF-IDF and n-grams detect changing vocabulary and rhetorical patterns, while stylometry captures the evolution of formal style across periods. The network analysis situates those linguistic features in a social structure, enabling historical interpretation beyond text-only analyses.

\subsection*{Technical Depth: Choices and Rationale}
The implementation reflects several decisions important for a reproducible and technically robust analysis.
\begin{itemize}
    \item \textbf{Document Unit for TF-IDF:} To capture period-specific vocabulary, the corpus is aggregated by period—this reduces noise at the single-letter level and sharpens the contrast between eras.
    \item \textbf{Preprocessing Rationale:} Using the WordNet lemmatizer and a custom stopword list aims to minimize archaic epistolary noise while preserving content-bearing tokens (e.g., place and person names). We opted to remove non-alphabetic characters to unify variant spellings (e.g., hyphenated forms), accepting the trade-off that editorial metadata sometimes uses abbreviations.
    \item \textbf{Scale and Efficiency:} The choice of JSON Lines and simple streaming preprocessing allows working with a large corpus on commodity hardware. Slurm/cluster usage minimizes the fragility of network-based harvesting and reduces run-time interruptions.
    \item \textbf{Sensitivity and Robustness:} Where appropriate, we cross-validate findings by comparing TF-IDF keywords with high-frequency n-grams and by manual inspection of outliers to identify OCR or metadata bleed.
\end{itemize}

% 5. Results
\section*{Results}
\addcontentsline{toc}{section}{Results}

The computational analysis of the Founding Fathers' correspondence reveals distinct temporal patterns across network structure, vocabulary usage, and stylistic evolution. These findings map closely to the major political and personal phases of the founders' lives, from the colonial era through the early republic to their final years.

\subsection*{Network Analysis: From Cliques to Centralization}
The visualization of the correspondence network across eight historical periods (Figures \ref{fig:network_part1} and \ref{fig:network_part2}) illustrates a fundamental structural transformation in the "Republic of Letters". Quantitative metrics for these networks are provided in Table \ref{tab:network_stats}.

\begin{table}[H]
\centering
\caption{Network Statistics by Historical Period}
\label{tab:network_stats}
\resizebox{\textwidth}{!}{%
\begin{tabular}{@{}lrrrl@{}}
\toprule
\textbf{Period} & \textbf{Nodes} & \textbf{Edges} & \textbf{Density} & \textbf{Most Central Figure (Degree)} \\ \midrule
Colonial & 1,454 & 2,211 & 0.0011 & Benjamin Franklin (922) \\
Revolutionary War & 5,079 & 9,755 & 0.0004 & George Washington (3,056) \\
Confederation & 2,396 & 4,454 & 0.0008 & George Washington (1,343) \\
Washington Presidency & 3,505 & 6,196 & 0.0005 & George Washington (2,398) \\
Adams Presidency & 1,994 & 3,388 & 0.0009 & John Adams (1,081) \\
Jefferson Presidency & 4,764 & 7,232 & 0.0003 & Thomas Jefferson (4,391) \\
Madison Presidency & 3,268 & 4,943 & 0.0005 & James Madison (2,217) \\
Post-Madison & 2,815 & 5,346 & 0.0007 & Thomas Jefferson (2,477) \\ \bottomrule
\end{tabular}%
}
\end{table}

\begin{figure}[H]
    \centering
    \begin{subfigure}[b]{0.45\textwidth}
        \includegraphics[width=\textwidth]{../network_Colonial.png}
        \caption{Colonial (1706--1774)}
    \end{subfigure}
    \hfill
    \begin{subfigure}[b]{0.45\textwidth}
        \includegraphics[width=\textwidth]{../network_Revolutionary_War.png}
        \caption{Revolutionary War (1775--1783)}
    \end{subfigure}
    
    \vspace{1em}
    
    \begin{subfigure}[b]{0.45\textwidth}
        \includegraphics[width=\textwidth]{../network_Confederation.png}
        \caption{Confederation (1784--1788)}
    \end{subfigure}
    \hfill
    \begin{subfigure}[b]{0.45\textwidth}
        \includegraphics[width=\textwidth]{../network_Washington_Presidency.png}
        \caption{Washington Presidency (1789--1797)}
    \end{subfigure}
    \caption{Evolution of the Correspondence Network: Part I. Note the transition from the fragmented Colonial network to the highly centralized "star topology" of the Revolutionary War and Washington Presidency.}
    \label{fig:network_part1}
\end{figure}

\begin{figure}[H]
    \centering
    \begin{subfigure}[b]{0.45\textwidth}
        \includegraphics[width=\textwidth]{../network_Adams_Presidency.png}
        \caption{Adams Presidency (1797--1801)}
    \end{subfigure}
    \hfill
    \begin{subfigure}[b]{0.45\textwidth}
        \includegraphics[width=\textwidth]{../network_Jefferson_Presidency.png}
        \caption{Jefferson Presidency (1801--1809)}
    \end{subfigure}
    
    \vspace{1em}
    
    \begin{subfigure}[b]{0.45\textwidth}
        \includegraphics[width=\textwidth]{../network_Madison_Presidency.png}
        \caption{Madison Presidency (1809--1817)}
    \end{subfigure}
    \hfill
    \begin{subfigure}[b]{0.45\textwidth}
        \includegraphics[width=\textwidth]{../network_Post-Madison.png}
        \caption{Post-Madison (1817--1836)}
    \end{subfigure}
    \caption{Evolution of the Correspondence Network: Part II. The network remains centralized around the executive but begins to diffuse in the Post-Madison era as the Founders retire.}
    \label{fig:network_part2}
\end{figure}

\textbf{The Colonial Era} is characterized by fragmentation. The graph shows distinct, loosely connected clusters representing regional elites (e.g., the Virginia planters vs. the Boston intellectuals) with no single dominant center. Benjamin Franklin holds the highest degree (922), reflecting his unique position as a colonial agent in London connecting disparate groups.

\textbf{The Revolutionary War} forces a dramatic centralization and expansion. The network size more than triples to 5,079 nodes. It adopts a "star topology" or "hub-and-spoke" model, with Washington as the undisputed central node (degree 3,056), mediating information between Congress, the Army, and the states. This is the visual signature of a command economy of information.

\textbf{The Confederation Period} shows a partial relaxation of this centrality, with new nodes of influence appearing (e.g., Jefferson and Adams in Europe), reflecting the diplomatic dispersion of the era.

\textbf{The Presidential Eras} (Washington through Madison) see the re-emergence of the executive branch as the gravitational center. However, subtle differences appear: the Washington network is tightly focused, while the Jeffersonian network appears slightly more distributed, potentially reflecting the Democratic-Republican emphasis on party machinery over pure executive authority. Notably, Jefferson's presidency exhibits the highest individual degree centrality in the entire corpus (4,391), surpassing even Washington's wartime command, indicating an immense personal investment in managing the administration via correspondence.

Finally, the \textbf{Post-Madison} network shows a "retirement diffusion." While key figures like Jefferson remain central due to their vast correspondence (degree 2,477), the network structure loosens, reflecting a shift from administrative command to intellectual exchange (the "University phase").

\subsection*{TF-IDF Analysis: The Evolution of Discourse}
Term Frequency-Inverse Document Frequency (TF-IDF \cite{salton1988}) analysis highlights the unique vocabulary defining each historical period. The results (see Appendix A) show a clear trajectory from personal and local concerns to national administration and finally to institutional legacy.

\begin{itemize}
    \item \textbf{Colonial (1706--1774):} The vocabulary is intensely local and personal. Terms like ``doeg'' (Doeg Run) and ``muddy hole'' reflect Washington's focus on land surveying and estate management. ``Cravenstreet'' points to Benjamin Franklin's London residence, while ``teedyuscung'' indicates specific, localized diplomatic engagements with Native American leaders, contrasting sharply with the abstract "Indian affairs" of later periods.
    \item \textbf{Revolutionary War (1775--1783):} The discourse shifts abruptly to military logistics. Top terms include ``middlebrook'', ``peekskill'', and ``pompton''—strategic encampments that dominated the writers' daily reality. The appearance of ``sartine'' and ``ternay'' reflects the existential reliance on the French alliance.
    \item \textbf{Confederation (1784--1788):} Vocabulary bifurcates between the domestic and the diplomatic. We see ``dogue'' and ``plowed'' (Washington returning to Mount Vernon) alongside ``calonne'' and ``thulemeier'' (Jefferson and Adams in Europe). This lexical split perfectly captures the era's lack of a central unifying focus.
    \item \textbf{Washington Presidency (1789--1797):} The language becomes administrative and institutional. Terms like ``assignats'' (French revolutionary currency) and ``genest'' (Citizen Genêt) mark the intrusion of global economic and political instability into the fragile new republic. Notably, artifacts such as ``hamiltonsecy'' appear, reflecting the formalization of titles and bureaucracy.
    \item \textbf{Adams Presidency (1797--1801):} The Quasi-War with France drives the vocabulary. ``Artillerists'', ``talleyrand'', and ``toussaint'' (L'Ouverture) rise to prominence, indicating a fixation on external threats and the Haitian Revolution.
    \item \textbf{Jefferson Presidency (1801--1809):} The focus shifts to continental expansion. ``Pichon'', ``turreau'', and ``yrujo'' (diplomats) are key, alongside ``natchitoches'', reflecting the administrative reality of the Louisiana Purchase.
    \item \textbf{Madison Presidency (1809--1817):} The War of 1812 dominates. ``Napoleon'' is the top term, but economic terms like ``merino'' (sheep) also appear, signaling the drive for domestic industrial independence (wool production) necessitated by the British blockade.
    \item \textbf{Post-Madison (1817--1836):} The vocabulary turns remarkably academic. Terms like ``dormitory'', ``bursar'', ``dunglison'' (a professor recruited by Jefferson), and ``bonnycastle'' dominate. The "Republic of Letters" becomes a literal university project.
\end{itemize}

\subsection*{Stylometric Analysis: Complexity and Richness}
Beyond vocabulary, the structure of the language evolves significantly (see Appendix B, Table \ref{tab:stylo}).
\begin{itemize}
    \item \textbf{Syntactic Complexity:} Average sentence length jumps from 18.8 words in the Colonial period to 26.4 words during the Revolutionary War. This 40\% increase suggests that the exigencies of war and statecraft required a more complex, qualified, and precise mode of expression than private business correspondence.
    \item \textbf{Lexical Richness:} Yule's K, a measure of vocabulary diversity, peaks at 119.5 during the Washington Presidency. This is the highest value in the entire corpus. It suggests that the task of inventing the American presidency required an unprecedented expansion of the political lexicon—the Founders were literally finding new words to describe a new form of government.
\end{itemize}

\subsection*{N-gram Analysis: Rhetorical Patterns}
N-gram analysis (Appendix B) corroborates the TF-IDF findings while revealing structural changes in language.
\begin{itemize}
    \item \textbf{The Disappearance of the Local:} In the Colonial period, top bigrams are specific and local, such as ``muddy hole'' (a farm on Washington's estate) and ``doeg run''. These vanish entirely in later periods, replaced by abstract political entities.
    \item \textbf{The Rise of the Nation:} The bigram ``united state'' is virtually absent in the Colonial period but becomes the dominant political entity from the Revolutionary War onwards.
    \item \textbf{Institutionalization:} The Revolutionary War is characterized by ``head quarter'', ``court martial'', and ``commander chief''—institutional terms that replace the personal salutations of the earlier era.
\end{itemize}
% 6. Critical Evaluation
\section*{Critical Evaluation}
\addcontentsline{toc}{section}{Critical Evaluation}
% Critically evaluate the model(s)/method(s) and your approach. 
% This should involve critical analysis of the adequacy of the modelling (e.g., what are the assumptions the models rely on, and do they all hold; are there factors or biases that might invalidate conclusions drawn, etc.), as well as comparison with external data and/or published research on the topic.
While the computational approach provides valuable macro-level insights, a critical evaluation reveals both the strengths and limitations of the methodology.

Following the Lecture 5 guidance for a summative assignment, the project begins with a clearly articulated historical question and chooses computational techniques that map to historical constructs (topic, formality, network influence) rather than starting with the closest available algorithm and shaping the question to that method.

\subsection*{Methodological Assumptions and Limitations}
The models rely on several key assumptions. First, Network Analysis assumes that the frequency of correspondence equates to the strength of a relationship. This ignores the qualitative nature of the letters; a single long, intimate letter may signify a stronger bond than ten brief administrative notes. Second, TF-IDF assumes that term frequency correlates with thematic importance. However, in 18th-century epistolary style, formulaic politeness (e.g., "your most obedient humble servant") is highly frequent but semantically empty. While we mitigated this with a custom stopword list, some noise inevitably remains.

Furthermore, the dataset is subject to \textbf{survival bias}, as not all historical correspondence has been preserved, and \textbf{selection bias}, as Founders Online focuses on prominent figures ("Great Man" history). As noted in the lecture material regarding n-gram assumptions, \textbf{OCR errors and metadata artifacts} significantly impact results. For instance, the appearance of ``hamiltonsecy'' in the TF-IDF top terms reveals that editorial annotations (e.g., "Hamilton, Secy.") were not fully separated from the body text during the scraping process. Similarly, spelling variations (e.g., ``Doeg Run'' vs. ``Dogue Run'') fragment the counts for single concepts, a common issue in pre-standardized 18th-century English that simple lemmatization cannot fully resolve.

\subsection*{Comparison with Historical Narratives}
Despite these limitations, the computational results align remarkably well with established historical narratives, such as Gordon Wood's account of the early republic \cite{wood2009}, validating the "distant reading" approach.
\subsection*{Synthesis of Methods}
By employing multiple methods, we can observe how structural power correlates with linguistic change. The Network Analysis shows a centralization of power around Washington during the war and presidency. The tripling of the network size during the Revolutionary War (1,454 to 5,079 nodes) coincides perfectly with the 40\% increase in sentence length, suggesting that managing a larger, more complex network required more complex linguistic structures. This structural shift is mirrored in the TF-IDF results, which transition from diverse local terms to a unified, administrative vocabulary. However, the methods also diverge in revealing ways: while Stylometry shows a relatively constant sentence length after the Revolutionary War spike, the vocabulary richness (Yule's K) peaks specifically during the Washington presidency. This suggests that while the \textit{syntactic complexity} of the Founders' language stabilized, the \textit{lexical diversity} required to build a nation increased significantly. They were not just writing longer sentences; they were deploying a wider array of concepts to define the new state. This multi-modal approach provides a more nuanced view than any single method could offer.
\begin{itemize}
    \item \textbf{The Professionalization of Politics:} The network shift from a fragmented social graph to a centralized "hub-and-spoke" model around the President reflects the formalization of the executive branch.
    \item \textbf{The Era of Good Feelings:} The Post-Madison vocabulary, focused on "university" and "education," aligns with the historical understanding of Jefferson and Madison's retirement years, where they turned from active politics to legacy-building.
\end{itemize}
This alignment suggests that while the models may be noisy at the micro-level (individual words), they are robust at the macro-level (historical trends).

\subsection*{Synthesis of Methods}
By employing multiple methods, we can observe how structural power correlates with linguistic change. The Network Analysis shows a centralization of power around Washington during the war and presidency. This structural shift is mirrored in the TF-IDF results, which transition from diverse local terms to a unified, administrative vocabulary. However, the methods also diverge: while Stylometry shows a relatively constant sentence length, the vocabulary richness (Yule's K) peaks during the Washington presidency, suggesting that while the \textit{complexity} of sentence structure remained stable, the \textit{lexical diversity} required to build a nation increased significantly. This multi-modal approach provides a more nuanced view than any single method could offer.

\subsection*{Evaluation Methodology and Validation}
To satisfy the assignment's requirement for critical evaluation and to validate claims derived from computational models, several evaluation strategies were used:
\begin{itemize}
    \item \textbf{Cross-method triangulation}: We checked whether terms identified by TF-IDF also appear as high-frequency n-grams in the same period. A high overlap increases confidence in thematic importance.
    \item \textbf{Manual verification} of top terms and sample letters: a historian-authored spot-check of the top TF-IDF terms in each period ensured that salient words correspond to the contexts inferred (e.g., military or administrative contexts).
    \item \textbf{Statistical checks}: Where appropriate, metrics such as average sentence length and Yule's K were compared across periods using simple descriptive statistics. Formal hypothesis testing (e.g., ANOVA or non-parametric equivalents) can be applied when explicitly testing period differences in sentence length or lexical richness.
    \item \textbf{Sensitivity analyses}: To test the stability of TF-IDF rankings, we re-ran TF-IDF with and without the custom stopword list and with a minimal lemmatization pipeline to measure term stability across preprocessing choices.
    \item \textbf{Metadata consistency checks}: Indexes for place names, person names, and editorial notes were checked for bleed-through (e.g., "hamiltonsecy"), and where these occur they were handled by either removal or separate analysis and flagged in the appendices.
    \item \textbf{No pre-trained models used without evaluation}: We did not rely on off-the-shelf pretrained language models for the corpus analysis because pretrained models often require domain adaptation; any future use of embeddings or pretrained classifiers would need explicit validation to show appropriateness for 18th/19th-century epistolary text.
\end{itemize}


These steps improve reproducibility and ensure that computational outputs are interpreted within a historically meaningful frame, as recommended in Lecture 5: begin with a plausible question, evaluate model assumptions, and ensure measured constructs are meaningful for the domain.

% Discuss what conclusions can be drawn from the model.
\section*{Conclusions}
\addcontentsline{toc}{section}{Conclusions}
This study demonstrates the utility of computational methods in enriching historical research. The analysis confirms that the Founding Fathers' correspondence patterns were not static but evolved dynamically in response to the changing political landscape. We observed a clear trajectory: from the personal and local concerns of the Colonial era, through the urgent military logistics of the Revolutionary War, to the complex administrative statecraft of the early Republic, and finally to a reflective focus on education and legacy in the Post-Madison years.

A key finding is the peak in vocabulary richness during the Washington Presidency, suggesting that the task of establishing a new national government required a uniquely complex and varied lexicon. These findings underscore the value of "distant reading" as a complement to traditional historical scholarship, offering a quantitative backbone to qualitative historical narratives and revealing the linguistic footprint of nation-building.

\subsection*{Further Work and Reflection}
To address remaining limitations and to deepen interpretation, a number of further steps could refine and extend this project:
\begin{itemize}
    \item \textbf{Topic modelling:} LDA (Latent Dirichlet Allocation) or related techniques could complement TF-IDF by deriving latent topics, testing whether aggregate topic prevalence changes across periods \cite{blei2003}.
    \item \textbf{Named Entity Recognition (NER):} Applying NER and linking entities to standard authorities (e.g., VIAF identifiers) would enable more accurate tracking of individuals and places across periods and reduce the effect of spelling variants.
    \item \textbf{Formal hypothesis testing:} Explicit testing—for example, whether sentence length distributions differ significantly between periods—would strengthen claims about stylistic change.
    \item \textbf{Improved OCR and spelling normalization:} Preprocessing improvements could reduce term fragmentation (e.g., Doeg/Dogue) and editorial artifacts (e.g., hamitonsecy). This includes custom spelling unification or editorial cleaning.
    \item \textbf{Authorship attribution and contextualization:} Applying supervised classification to determine if stylistic changes are driven by authorship differences or by broader social changes would refine causal claims.
\end{itemize}

\newpage
\section*{References}
\addcontentsline{toc}{section}{References}
% Use BibTeX, BibLaTeX, or manual bibliography here. Example below is manual entries.
\begin{thebibliography}{9}
\bibitem{salton1988} Salton, G., \& Buckley, C. (1988). Term-weighting approaches in automatic text retrieval. Information Processing \& Management, 24(5), 513--523.
\bibitem{wood2009} Wood, G. S. (2009). Empire of Liberty: A History of the Early Republic, 1789-1815. Oxford University Press.
\bibitem{blei2003} Blei, D.M., Ng, A.Y., \& Jordan, M.I. (2003). Latent Dirichlet Allocation. Journal of Machine Learning Research, 3, 993--1022.
\bibitem{yule1944} Yule, G. U. (1944). The Statistical Study of Literary Vocabulary. (Cambridge University Press).
\bibitem{bird2009} Bird, S., Klein, E., \& Loper, E. (2009). Natural Language Processing with Python. O'Reilly Media.
\bibitem{hagberg2008} Hagberg, A., Schult, D., \& Swart, P. (2008). Exploring network structure, dynamics, and function using NetworkX. In Proceedings of the 7th Python in Science Conference (SciPy2008).
\bibitem{nltk} Bird, S., Loper, E., \& Klein, E. (2009). Natural Language Toolkit (NLTK). http://www.nltk.org
\end{thebibliography}

\newpage
\appendix
\section{TF-IDF Results}
\begin{table}[H]
\centering
\caption{Top 10 TF-IDF Terms by Historical Period}
\label{tab:tfidf}
\resizebox{\textwidth}{!}{%
\begin{tabular}{@{}llllllll@{}}
\toprule
\textbf{Colonial} & \textbf{Rev. War} & \textbf{Confederation} & \textbf{Washington} & \textbf{Adams} & \textbf{Jefferson} & \textbf{Madison} & \textbf{Post-Madison} \\ \midrule
doeg & artaud & dogue & hamiltonsecy & subdistricts & pichon & napoleon & dormitory \\
ageneral & middlebrook & williamos & ustates & artillerists & cevallos & batture & dunglison \\
eund & zullen & deral & assignats & mchenry & turreau & merino & montezillo \\
dismd & tyonderoga & mansn & hamiltonsecretary & dittovice & reibelt & correa & bursar \\
patd & sartine & sjl & jaudenes & rivardi & osage & gothenburg & montpr \\
empld & eenige & tabacs & ischem & talleyrand & yrujo & ticknor & brockenbrough \\
dind & peekskill & shipton & chiappe & bewell & ustates & bonaparte & raggi \\
waterson & pompton & deslon & clingman & dittojohn & polygraph & bankhead & capitels \\
cravenstreet & hazen & grd & whitting & subdistrict & gallatin & nonag & ticknor \\
magowan & billingsport & doradour & genest & tousard & gavino & dallas & bonnycastle \\ \bottomrule
\end{tabular}%
}
\end{table}

\section{N-gram Results}
\begin{table}[H]
\centering
\caption{Top 10 Bigrams by Historical Period}
\label{tab:bigrams}
\resizebox{\textwidth}{!}{%
\begin{tabular}{@{}llllllll@{}}
\toprule
\textbf{Colonial} & \textbf{Rev. War} & \textbf{Confederation} & \textbf{Washington} & \textbf{Adams} & \textbf{Jefferson} & \textbf{Madison} & \textbf{Post-Madison} \\ \midrule
new york & head quarter & united state & united state & united state & united state & united state & united state \\
home day & new york & new york & new york & new york & new york & new york & new york \\
little wind & united state & noon night & secretary treasury & secretary war & new orleans & john adam & john adam \\
benjamin franklin & respect excellency & morning noon & secretary state & mount vernon & department state & respect jefferson & james madison \\
george washington & que vous & muddy hole & president united & john adam & james madison & esteem respect & respect jefferson \\
every thing & every thing & mount vernon & house representative & major general & respect jefferson & accept assurance & accept assurance \\
went away & beg leave & dogue run & obedt servt & general hamilton & beg leave & every thing & esteem respect \\
muddy hole & servt washington & thermometer morning & servt jefferson & obedt servt & every thing & jefferson monticello & jefferson monticello \\
mount vernon & obedt servt & que vous & treasury department & war department & president united & james madison & thomas jefferson \\
fort cumberland & west point & every thing & every thing & president united & secretary state & beg leave & university virginia \\ \bottomrule
\end{tabular}%
}
\end{table}

\section{Stylometric Results}
\begin{table}[H]
\centering
\caption{Stylometric Analysis by Period}
\label{tab:stylo}
\begin{tabular}{@{}lcccc@{}}
\toprule
\textbf{Period} & \textbf{Letters} & \textbf{Avg Word Len} & \textbf{Avg Sent Len} & \textbf{Yule's K (Richness)} \\ \midrule
Colonial & 16,151 & 4.39 & 18.81 & 93.11 \\
Revolutionary War & 48,188 & 4.46 & 26.43 & 96.38 \\
Confederation & 17,810 & 4.43 & 21.72 & 97.44 \\
Washington Presidency & 27,156 & 4.51 & 23.02 & 119.51 \\
Adams Presidency & 13,570 & 4.49 & 23.55 & 109.19 \\
Jefferson Presidency & 29,499 & 4.51 & 24.02 & 110.65 \\
Madison Presidency & 15,477 & 4.48 & 26.53 & 106.81 \\
Post-Madison & 15,445 & 4.47 & 26.14 & 109.56 \\ \bottomrule
\end{tabular}
\end{table}

\end{document}
